\documentclass[xcolor={dvipsnames,table},14pt,aspectratio=169]{beamer}
\usepackage[portuguese]{babel}
\usepackage[T1]{fontenc}
\usepackage{graphicx}
\usepackage[utf8x]{inputenc}
\usepackage{verbatim}
\usepackage{tikz}

%-----------------------------------------------------------------------------------%

% ------------------ Pacotes para a tabela -----------------------------------------%

\usepackage{tabularx}                                                               
\usepackage{multirow}
\usepackage{rotating}
\usepackage{latexsym}
\usepackage{subfigure}
\usepackage{amsmath}
\usepackage{amsfonts}
\usepackage{amssymb}
\usepackage{amsthm}	
\usepackage{fancyhdr}
\usepackage{breakurl}
\usepackage{hyperref}
%\usepackage{mathtools}

%\DeclarePairedDelimiter\ceil{\lceil}{\rceil}
%\DeclarePairedDelimiter\floor{\lfloor}{\rfloor}

%-----------------------------------------------------------------------------------%

\mode<presentation>{
	\usetheme{Madrid} 
	\useinnertheme{rounded}
	\usecolortheme[named=purple]{structure}
	\usefonttheme[onlymath]{serif} 	% fonte modo matematico
}

\title [Engenharia de Software]{ChatGPT for Learning HCI Techniques: A Case Study on Interviews for Personas - mais uma alteracao}
\author[Simone Santos]{Apresentado por: Simone de Oliveira Santos\\Programa de Pós-Graduação em Ciência da Computação}
\institute[UECE]{ \includegraphics[scale=1]{figs/logo-uece}}
\date{}


%---------------------------------------------------------------------------------------------------------------------

\begin{document}
	
	%-----------------------------------------------------------------------------------------------------------------------------
	
	\begin{frame}
		
		\titlepage
		
	\end{frame}
	
	%-----------------------------------------------------------------------------------------------------------------------------
	
	\begin {frame}{Preliminares}
	
	\begin{itemize}
		\item Publicado no IEEE Transactions on Learning Technologies
		\item Qualis A1
		\item Ano de publicação 2024
	\end{itemize}
	
	
	\end {frame}
	
	%-----------------------------------------------------------------------------------------------------------------------------
	
	\begin {frame}{Preliminares}
	
	\begin{figure}
		\centering
		\includegraphics[width=1\linewidth]{figs/capa}
		\label{fig:capa}
	\end{figure}
	
	\end {frame}
	
	%-----------------------------------------------------------------------------------------------------------------------------  
	%                Introdução
	%-----------------------------------------------------------------------------------------------------------------------------  
	
	
	
	\begin{frame}{Motivação} 
		
		\begin{itemize}
			
			\item Antes de interagir com usuários reais, desenvolvedores devem entender de IHC para \textbf{não desperdiçar a paciência e disponibilidade dos usuários}.
			
			\item Especialistas no campos de IHC passam por \textbf{treinamento} em várias técnicas.
			
			\item Necessidade de analisar usuários em potencial para os sistemas que são produzidos.
			
		\end{itemize}
		
	\end{frame}
	
	
	
	\begin{frame}{Motivação}
		
		\begin{itemize}
			\item Conduzir entrevistas \textbf{é desafiador}.
			
			\item Recrutar participantes, definir questões efetivas para obter respostas úteis é imperativo.
			
			\item Entrevistas de baixa qualidade comprometem a análise.
			
			\item Técnicas como a criação de \textbf{Personas} somente funciona através de uma \textbf{boa coleta de dados e análise do contexto de uso}.
		\end{itemize}
		
	\end{frame}
	
	%%%-----------------------------------------------------------------------------------------------------------------------------
	
	\begin{frame}{Motivação}
		
		\begin{itemize}
			
			\item O surgimento das IA generativas e o seu uso como chatbots treinados (ChatGPT) oferece uma oportunidade uso para aprendizado.
			
			\item Estudos de caso recentes:
			\begin{itemize}
				\item aquisição de habilidades pedagógicas
				
				\item efeitos na motivação de estudantes e desempenho na aprendizagem
				
				\item sala de aula invertida
			\end{itemize} 
			
		\end{itemize}
		
		
		\end {frame}
		
		
		
		\begin{frame}{Lacuna identificada}
			
			\begin{block}{}
				Uso de ChatGPT para o aprendizado de técnicas para o design centrado no usuário.
			\end{block}
			
			\begin{flushleft}
				Técnica alvo: Criação de \textbf{Personas}
			\end{flushleft}
			
		\end{frame}
		
		
		
		\begin{frame}{Personas}
			
			\begin{block}{}
				No contexto de IHC, a criação de Personas é uma técnica usada para representar usuários reais e potenciais usuários para aplicações.
			\end{block}
			
			Uma persona é:
			\begin{itemize}
				\item fictícia
				\item uma representação realística de um usuário em potencial
				\item exemplifica atitudes associadas ao contexto dos usuários em relação a um produto específico.
			\end{itemize}
			
		\end{frame}
		
		
		
		\begin{frame}{Problema}
			
			\begin{flushleft}
				No contexto educacional é difícil treinar estudantes para conduzir atividades de IHC em contexto real.
			\end{flushleft}
			
			Fatores limitantes:
			\begin{itemize}
				\item acesso limitado à usuários reais
				\item limitações de tempo
				\item \textbf{falta de prática e habilidades em entrevistas}
			\end{itemize}
			
		\end{frame}
		
		
		
		\begin{frame}{Problema}
			
			\begin{flushleft}
				Para melhor entender a percepção dos problemas enfrentados, os pesquisadores conduziram um survey com estudantes de graduação.
			\end{flushleft}
			
			Achados:
			\begin{itemize}
				\item Dificuldades de recrutamento
				
				\item Falta de variabilidade de usuários
				
				\item Definir questões de qualidade
			\end{itemize}
			
		\end{frame}
		
		
		\begin{frame}{Objetivo}
			
			\begin{block}{}
				O objetivo da pesquisa é estudar e validar se o ChatGPT, propriamente configurado e parametrizado, pode gerar informações fictícias mas de boa qualidade para ser usado como material de treinamento em IHC.
			\end{block}
			
		\end{frame}
		
		
		\begin{frame}{Proposta}
			
			\begin{flushleft}
				A questão central da pesquisa é se um chatbot baseado em LLM pode servir como uma \textbf{ferramenta educacional de treinamento} para atividades de IHC.
			\end{flushleft}
			
			\begin{block}{Hipótese levantada}
				ChatGPT pode melhorar a experiência educacional pela simulação de interações de conversa, e consequentemente, ajudar estudantes a aprender.
			\end{block}
			
		\end{frame}
		
		
		
		\begin{frame}{Questões de pesquisa}
			
			\begin{enumerate}
				\item O ChatGPT pode representar um usuário fictício e crível através de uma configuração apropriada e um prompt único?
				
				\item É possível usar esses usuários fictícios para gerar um conjunto de entrevistas de alta qualidade que sirva como entrada válida para criar personas realísticas?
			\end{enumerate}
			
		\end{frame}
		
		
		\begin{frame}{Proposta}
			
			\begin{block}{}
				\textbf{Principal requisito}: que os resultados das entrevistas pudessem ser comparáveis, assim, foi usada a entrevista estruturada.
			\end{block}
			
			\begin{block}{}
				\textbf{Principal desafio}: operacionalizar o conceitos de ``usuário crível'' e ``entrevistas de qualidade''.
			\end{block}
			
			
		\end{frame}
		
		
		
		\begin{frame}{Proposta}
			
			\begin{block}{}
				Como cenário de experimentação: foi usado o contexto e os dados de um projeto real que objetivava melhorar a intranet da universidade.
			\end{block}
			
			\begin{flushleft}
				Este projeto foi desenvolvido de acordo com o processo centrado no usuário com criação de 8 personas.
			\end{flushleft}
			
		\end{frame}
		
		
		
		\begin{frame}{Metodologia}
			
			\begin{flushleft}
				A pesquisa foi estruturada com 4 grupos distintos:
			\end{flushleft}
			\begin{description}
				\item[Supervisor] Coordenador dos times
				
				\item[Especialistas UX] Criam os prompts e entrevistas baseado no projeto original. Foram criados 40 usuários fictícios.
				
				\item[Profissionais UX] Conhecedores do projeto original.
				
				\item[Professores IHC] Avaliam a qualidade do material e criam personas baseado nas entrevistas geradas.
			\end{description}
			
		\end{frame} 
		
		
		
		\begin{frame}
			
			\begin{figure}
				\centering
				\includegraphics[width=0.5\linewidth]{figs/metodologia}
			\end{figure}
			
			
		\end{frame} 
		
		%%
		%%%-----------------------------------------------------------------------------------------------------------------------------
		
		\begin{frame}{Metodologia}
			
			\begin{figure}
				\centering
				\includegraphics[width=0.7\linewidth]{figs/metodologia-topo}
			\end{figure}
			
			
		\end{frame} 
		
		
		
		\begin{frame}{Metodologia}
			
			\begin{figure}
				\centering
				\includegraphics[width=0.7\linewidth]{figs/metodologia-base}
			\end{figure}
			
			
		\end{frame} 
		
		
		\begin{frame}{Metodologia}
			
			\begin{flushleft}
				O objetivo da análise foi determinar se as respostas geradas pelo ChatGPT poderia ser \textbf{corretamente associadas com as personas originais} e se foram \textbf{informativas o suficiente} para apoiar a criação de novas personas.
			\end{flushleft}
			
		\end{frame}
		%------------------------------------------------------------
		%------------------------------------------------------------
		
		\begin{frame}{Medida de sucesso} 
			
			\begin{flushleft}
				Recapitulando as questões de pesquisa:
			\end{flushleft}
			
			\begin{itemize}
				\item[RQ1] O ChatGPT pode representar um usuário fictício e crível através de uma configuração apropriada e um prompt único?
				
				\item[RQ2] É possível usar esses usuários fictícios para gerar um conjunto de entrevistas de alta qualidade que sirva como entrada válida para criar personas realísticas?
			\end{itemize}
			
		\end{frame}
		
		
		\begin{frame}{Medida de sucesso} 
			
			\begin{itemize}
				\item[RQ1] Os profissionais classificam as respostas geradas com relação às 8 personas de referência, se os usuários fictícios são críveis.
				
				\item[RQ2] A qualidade do material foi medida de duas formas:
				\begin{description}
					\item[Profissionais] a qualidade do material do ponto de vista técnico, se representa o domínio do projeto original.
					\item[Professores] a qualidade do material do ponto de vista educacional, se é útil para criar personas.
				\end{description}
				
				
			\end{itemize}
			
			
		\end{frame}
		
		
		\begin{frame}{Template para criação de personas} 
			
			\begin{figure}
				\centering
				\includegraphics[width=0.7\linewidth]{figs/personas}
			\end{figure}
			
		\end{frame}
		
		
		
		\begin{frame}{Teste de operação do ChatGPT}	
			
			\begin{figure}
				\centering
				\includegraphics[width=0.5\linewidth]{figs/testeChat}
			\end{figure}
			
		\end{frame}
		
		
		
		\begin{frame}{Proposta de prompt}	
			
			\begin{figure}
				\centering
				\includegraphics[width=0.7\linewidth]{figs/perguntas}
			\end{figure}
			
		\end{frame}
		
		
		
		\begin{frame}{Medida de sucesso}
			
			\begin{itemize}
				\item Ao final do procedimento, os profissionais e professores devem responder a um questionário sobre a qualidade das entrevistas.
				
				\item Questões sobre formato da entrevista, informações relevantes, coerência das respostas, utilidade educacional, grau de simulação e ética.
			\end{itemize}
			
			
		\end{frame}
		
		%----------------------------------------------------------------
		%----------------------------------------------------------------
		
		\begin{frame}{Resultados - RQ1 ``believability''}
			
			\begin{itemize}
				\item De forma geral, os itens da entrevista \textbf{foram melhor avaliados pelos profissionais} do que pelos professores.
				
				\item Os profissionais já tinham conhecimento das personas do projeto original e apenas indicariam se a entrevista representava bem.
				
				\item Os professores precisavam das entrevistas para criar personas e foram mais críticos.
			\end{itemize}
			
			
		\end{frame}
		
		
		\begin{frame}{Resultados - RQ1 ``believability''}
			
			\begin{flushleft}
				Principais achados da avaliação qualitativa:
			\end{flushleft}
			
			\begin{itemize}
				\item  Respostas repetitivas, superficiais, e não elaboravam as respostas.
				
				\item Inconsistência na linguagem: muito boa para um estudante internacional, por exemplo, e inconsistência de gênero (Espanhol).
				
				\item Informações incompletas para criação de personas.
			\end{itemize}
			
			
		\end{frame}
		
		
		\begin{frame}{Resultados - RQ1 ``believability''}
			
			\begin{itemize}
				\item Apesar dos problemas, \textbf{a resposta à RQ1 foi positiva}.
				
				\item Considerando cada entrevista separadamente, ambos \textbf{consideraram os usuários fictícios críveis} para o contexto do sistema.
			\end{itemize}
			
			
		\end{frame}
		
		
		
		\begin{frame}{Resultados - RQ2 ``quality''}
			
			\begin{itemize}
				\item Os profissionais \textbf{relacionaram todas as entrevistas} com uma persona do projeto original, validando o processo de criação de prompts para gerar entrevistas com informações úteis.
				
				\item Os professores \textbf{foram capazes de gerar personas}, com exceção de uma, que casavam com as originais.
			\end{itemize}
			
		\end{frame}
		
		\begin{frame}{Resultados - RQ2 ``quality''}
			
			\begin{itemize}
				\item A resposta à pergunta foi considerada \textbf{positiva}.
				
				\item A maior limitação considerada foi a repetição do conteúdo devido à limitação de respostas do chat para uma melhor consistência e falta de elaboração das respostas.
			\end{itemize}
			
			
		\end{frame}
		
		
		
		\begin{frame}{Discussões - Limitações}
			
			\begin{itemize}
				\item O prompt fornecia algumas alternativas para o chat assumir na criação de usuário, e o chat não elaborava na escolha e não randomizada nas escolhas, repetindo escolhas.
				
				\item O chat tende a escolher o gênero masculino (39 de 40).
				
				\item Possível viés dos professores que conhecem o tópico do contexto e conhecem os papéis do projeto.
			\end{itemize}     
			
		\end{frame}
		
		
		\begin{frame}{Discussões - Recomendações}
			
			
			\begin{itemize}
				\item Atenção ao trabalhar com mais de uma\textbf{ versão do ChatGPT} ou similares pois podem apresentar comportamentos diferentes.
				
				\item Em entrevistas, \textbf{simplificar e reduzir o número de perguntas}, pois ele pode juntar todas as respostas em uma só e depois repetia as respostas nas demais perguntas.
				
				\item \textbf{Não há garantia de diversidade} quando são fornecidas escolhas. É preciso explicitamente indicar variedade.
			\end{itemize}
			
			
			\end {frame}
			
			
			\begin{frame}{Conclusões - Lições aprendidas}
				
				
				\begin{itemize}
					\item Entendimento do funcionamento de um chatbot baseado em LLM para fornecer boas instruções.
					
					\item Fornecer flexibilidade suficiente para evitar respostas repetidas. Tentativa e erro foi a forma encontrada para resolver a questão.
					
				\end{itemize}
				
				\begin{block}{}
					Passo interessante para integrar tecnologias baseadas em LLM no \textbf{treinamento de técnicas de IHC}, sem intenção de substituir os usuários reais.
				\end{block}
				
				\end {frame}
				
				
				\begin{frame}
					
					\titlepage
					
				\end{frame}
				
				%-----------------------------------------------------------------------------------------------------------------------------
				
			\end{document}
